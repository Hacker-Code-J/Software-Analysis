% Chapter II. Greybox Fuzzing and Concolic Testing

\section{Greybox Fuzzing}
How to Generate Automatic Test Cases?
\subsection{Setup}
\begin{itemize}
	\item $\cP$: Program under test (PUT)
	\item Program execution(semantics): \[
	\llbracket\mathcal{P}\rrbracket:\Sigma^*\to\cR
	\] \begin{itemize}
		\item $\Sigma$: input characters, e.g., ASCII Code
		\item $\cR$: execution results
	\end{itemize}
	\item Test Oracle: \[
	\oracle:\Sigma^*\times\cR\to\set{\bot,\top}
	\]
	\begin{itemize}
		\item $\top$: the program has run correctly (expected outcome)
		\item $\bot$: the program has run incorrectly (unexpected outcome)
		\item E.g., ``crash oracle'', reference implementation, etc. 
	\end{itemize}
\end{itemize}

\subsection{Random Fuzzing}
\begin{algorithm}\DontPrintSemicolon
	\caption{Random Fuzzing}
	\Procedure{\textsc{\upshape RandomFuzzer}($\cP$)}{
		$\text{bugs}\gets\emptyset$\;
	 	\While{\upshape time budget expires}{
 			$\text{in}\gets\textsc{Sample}(\Sigma^*)$\;
 			$\text{res}\gets\llbracket\cP\rrbracket(\text{in})$\;
 			\If{\upshape$\oracle(\text{in},\text{res})=\bot$}{
 				$\text{bugs}\gets\text{bugs}\cup\set{\text{in}}$\;
 			}
 		}
 		\Return $\text{bugs}$\;
	}
\end{algorithm}
	
\newpage
\subsection{Limitation}
\begin{itemize}
	\item Programs typically expect inputs in specific language $(L\subseteq \Sigma^*)$
	\begin{itemize}
		\item e.g., web browsers, image processors, compliers, etc.
	\end{itemize}
	\item Random inputs are unlikely to exercise deep program paths
\end{itemize}

\subsection{(Structural) Code Coverage}
\begin{lstlisting}[style=C]
int foo (int x, int y) {
	int z = 0;
	if (x > 3 && y < 6) {
		z = x;
	}
	return z; 
}
\end{lstlisting}

\section{Concolic Testing}